\documentclass{article}

\usepackage[utf8]{inputenc}
\usepackage{amsmath}
\usepackage{caption}
\usepackage{graphicx}
\usepackage{wrapfig}
\graphicspath{{./img/}}
\DeclareGraphicsExtensions{.pdf,.png,.jpg}

\begin{document}
\title{Paraxial Approximation for Stormtroopers}

\author{Darth Vader}

% hide date and abstract heading
\date{}

\maketitle

\begin{abstract}
We propose a new method to approximate the trajectory of a projectile expelled
from a firearm. Empirical tests show that this new method is sufficient for
targets located at a maximal distance of a single meter.
\end{abstract}

\begin{wrapfigure}{r}{0.3\textwidth}
	\vspace{-15pt}
	\captionsetup{
	        format=plain,
	        font=footnotesize
	}
	\begin{center}
	\def\svgwidth{\linewidth}
	\input{img/paraxialapprox.pdf_tex}
	\caption{Geometrische Interpretation der Kleinwinkelnäherung. Für kleine
	Winkel ist $y\approx \theta$.}
	\label{fig:paraxialapprox}
	\end{center}
	\vspace{-20pt}
\end{wrapfigure}

Die \textit{paraxiale Approximation} ist lineare Approximation der Trajektorie,
die besonders für kleine Winkel geeignet ist.

% Geometrische/Anschauliche Definition
Anschaulich besagt diese, dass für kleine Winkel $|\theta|$, die Bogenlänge auf
dem Einheitskreis kaum von der Höhe $h$ des Steigungsdreiecks abweicht.
Abbildung~\ref{fig:paraxialapprox} demonstriert diesen Zusammenhang. Es gilt
also $\sin(x) \approx x$ für $|x| \ll 1$.

% Zusammenhang mit der Taylorreihe
Analytisch erhält man diesen Zusammenhang aus der Taylorreihe
$\sum_{n=0}^\infty  \frac{f^{(n)}(a)}{n!} (x-a)^n$, welche für die
Sinusfunktion und die Entwicklungsstelle $a=0$ zu $$
sin(x) = \sum_{n=0}^\infty (-1)^n \frac{x^{2n+1}}{(2n+1)!} = x - \frac{x^3}{6} + \frac{x^5}{120} - \cdots
$$ auswertet. Insbesondere sei angemerkt, dass alle Monome mit geradem
Exponenten entfallen, da $(\pm sin(a := 0))' = 0$. Die Kleinwinkelnäherung ergibt
sich, indem in der Taylorentwicklung alle Monome mit einem Grad $g > 1$
vernachlässigt werden.

% Beliebige präzision
Mittels des Quotientenkriterums folgt
\begin{align*}
&\lim_{n \to \infty} \left| (-1)^{n+1} \frac{x^{2(n+1)+1}}{(2(n+1)+1)!} \bigg/ (-1)^n \frac{x^{2n+1}}{(2n+1)!} \right|\\
                   =& \lim_{n \to \infty} \left| \frac{(2n+1)!}{(2n+3)!} x^2\right|\\
		   =& \lim_{n \to \infty} \left| \frac{1}{(2n+2)(2n+3)} x^2\right|\\
		   =&\enskip 0\text{.} %\blacksquare
\end{align*}
Der Konvergenzradius der Taylorreihe für $\sin(x)$ ist also unendlich. Dies
bedeutet, dass $\sin(x)$ ausgehend von jeder Entwicklungsstelle $a$ beliebig
durch Hinzunahme von Monomen höheren Grades in die Taylorreihe approximiert
werden kann. 

\end{document}
